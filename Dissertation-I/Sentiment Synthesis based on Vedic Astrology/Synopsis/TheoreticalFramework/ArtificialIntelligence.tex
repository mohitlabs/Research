Artificial Intelligence (AI) refers to the simulation of human intelligence in machines that are programmed to perform tasks that normally require human intelligence such as learning, problem-solving, decision-making, perception, language understanding, and more\cite{ISSN-2456-2165}. AI systems use algorithms and statistical models to analyze data, recognize patterns, and make predictions, without explicit instructions from human operators.

\subsubsection{Limitations of AI}
Although artificial intelligence (AI) has made significant advancements in recent years, there are still some limitations to the technology. Some of the limitations are:
\begin{itemize}
	\item \textbf{Lack of Common Sense:} AI systems lack the common sense that humans have, which can make it difficult for them to understand complex situations and make appropriate decisions.
	\item \textbf{Limited Creativity:} AI systems are designed to operate within the parameters set by their algorithms and data, which limits their ability to generate truly creative solutions or ideas.
	\item \textbf{Lack of Emotional Intelligence:} AI systems are not capable of experiencing emotions, which limits their ability to understand and respond to emotional cues in human interactions.
\end{itemize}
Limitations of AI highlight the need for continued research and development to address these issues and improve the capabilities of these systems.
