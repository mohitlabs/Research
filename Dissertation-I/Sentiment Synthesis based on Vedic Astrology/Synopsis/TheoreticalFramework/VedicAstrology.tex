Astrology is like the snapshot of our unconscious mind. The planetary vibrations reflected or refracted along with solar radiations to the earth are of varying intensities as per planetary distance, size, and movement in the solar system. These vibrations impact our sensory nerves, mental attitudes, and moods. Thus, it’s very likely that these planetary vibrations supply the energies to the body cells though our nerves. Since these vibrations differ in wavelength intensity and frequency as per the planetary properties and motion. These vibrations supply different sensory stimuli which impacts the human unconscious and personality at the time of birth\cite{article1}.

\subsubsection{Classification of Vedic Astrology}
Vedic astrology, also known as Jyotish, is an ancient system of astrology that originated in the Indian subcontinent. It is considered one of the oldest astrological systems in the world and has its roots in the Vedas, the ancient sacred texts of Hinduism.
Vedic astrology(Jyotish) is one of the most important limb(Vedanga) out of the total six limbs(Vedangas) found in the ancient Indian scriptures. The origins of Vedic astrology can be traced back to around 1500 BCE, during the late Vedic period. It is classified into the three major branches or disciplines known as the Siddhanta, Samhita and Hora. These branches provide different approaches and methods for studying and practicing astrology.

\begin{itemize}
	\item \textbf{Siddhanta:} Siddhanta deals with all the mathematical calculations of space \& time which is involed in the study of planets, stars, comets and contellations present in the space. It is also referred as the Astronomy in modern days.
	\item \textbf{Samhita:} Samhita, also known as Muhurtha, is the branch of Vedic astrology that deals with collective or mundane astrology. It focuses on predicting and analyzing events and phenomena on a broader scale, such as natural disasters, weather patterns, political developments, and societal events.
	\item \textbf{Hora:} Hora or horarian astrology is the branch of Vedic astrology that specifically deals with individual horoscopes or birth charts (Jataka).
\end{itemize}

There are many ancient texts and scriptures about all these three branches of the vedic astrology, that were written by Maharishies. But here, we are only considering the Surya Siddhanta\cite{SuryaSiddhanta, wiki:ss} and Brihat Parashara Hora Shastra\cite{BrihatParasharHoraShastraVol1, BrihatParasharHoraShastraVol2, wiki:bphs} which are the most popular texts that are used by most of the astrologers today for astrological predictions and consultations. We also discuss some of rules and priciples described by the dictums found in these texts.

\subsubsection{Properties of Planets}
The study of planets holds great significance in understanding an individual's horoscope and predicting future events. Vedic astrology recognizes nine celestial bodies as planets, including the Sun, Moon, Mercury, Venus, Mars, Jupiter, Saturn, Rahu, and Ketu. Each planet's placement, aspects, and interactions with other planets in an individual's birth chart play a crucial role in determining their personality traits, strengths, challenges, and life events according to Vedic astrology. Here's a brief introduction to each of these planets in the context of Vedic astrology:

\begin{itemize}
	\item \textbf{Sun (Surya):} The Sun represents the self, vitality, authority, and leadership. It symbolizes the soul and is considered the king among the planets.
	\item \textbf{Moon (Chandra):} The Moon represents emotions, mind, instincts, and nurturing qualities. It influences the individual's emotional well-being and intuition.
	\item \textbf{Mercury (Budha):} Mercury is associated with communication, intellect, logic, and adaptability. It governs speech, learning, and mental abilities.
	\item \textbf{Venus (Shukra):} Venus is the planet of love, beauty, art, and harmony. It governs relationships, attraction, creativity, and material pleasures.
	\item \textbf{Mars (Mangal):} Mars signifies energy, passion, courage, and assertion. It governs ambition, drive, physical strength, and competitiveness.
	\item \textbf{Jupiter (Guru):} Jupiter is considered the most benefic planet in Vedic astrology. It represents wisdom, knowledge, expansion, abundance, and spirituality.
	\item \textbf{Saturn (Shani):} Saturn is associated with discipline, responsibility, hard work, and karmic lessons. It teaches patience, endurance, and signifies limitations and life challenges.
	\item \textbf{Rahu:} Rahu is a shadow planet and represents material desires, obsession, illusions, and worldly attachments. It signifies ambition and can bring sudden changes.
	\item \textbf{Ketu:} Ketu is the other shadow planet and represents spirituality, detachment, mysticism, and karmic lessons. It signifies liberation and can bring unconventional experiences.
\end{itemize}

\subsubsection{Dashas \& Transits}
In Vedic astrology, dashas and transits are important techniques used to predict and analyze various aspects of a person's life based on the positions of planets and their movements at particular time periods after the birth of an individual. Chapter 53 of BPHS\cite{BrihatParasharHoraShastraVol1, BrihatParasharHoraShastraVol2, wiki:bphs} discusses about different effects of dashas of the different planets.

\begin{itemize}
	\item \textbf{Dashas:} Dashas are planetary periods that determine the major themes and influences in a person's life. They are calculated based on the position of the Moon at the time of birth. Each dasha is associated with a specific planet, and it represents a specific time period during which that planet's energy dominates. The most commonly used dasha system is the Vimshottari Dasha, which consists of a cycle of major and sub-periods of the planets.
	\item \textbf{Transits:} Transits, also known as `gochara', refer to the current positions of planets in relation to an individual's birth chart. They provide insights into the ongoing influences and events happening in a person's life. The movement of planets in the sky affects different areas of life and can trigger various experiences and opportunities. Astrologers analyze transits to understand the potential effects of planetary movements on different aspects of life, such as career, relationships, health, and finance. Table \ref{Table:table} represents the time required by all nine planets to complete one zodiac sign which is calculated in chapter 1, verse 29 to 34 of Surya Siddhanta\cite{SuryaSiddhanta, wiki:ss}.
\end{itemize}

\noindent
\begin{table}[H]
	\begin{tabularx}{\columnwidth}{|M|M|M|}
		\hline
		\textbf{Graha(Planet)} & \textbf{Angular Speed(\textdegree/Day)} & \textbf{Time For One Zodiac Sign} \\
		\hline
		Surya(Sun) & 1 & 1 Month \\
		\hline
		Chandra(Moon) & 13 & 2.25 Days \\
		\hline
		Brihaspati(Jupiter) & 1/12 & 1 Year \\
		\hline
		Shani(Saturn) & 1/30 & 2.5 Years \\
		\hline
		Budh(Mercury) & 1 & 1 Month \\
		\hline
		Shukra(Venus) & 1 & 1 Month \\
		\hline
		Mangal(Mars) & 2/3 & 1.5 Month \\
		\hline
		Rahu & -1/18 & 18 Months \\
		\hline
		Ketu & -1/18 & 18 Months \\
		\hline
	\end{tabularx}
	\caption{Time required by all planets to complete one zodiac sign}
	\label{Table:table}
\end{table}

Dashas and transits are used together to gain a comprehensive understanding of a person's life events and their timing. Dashas provide long-term trends and major life themes, while transits offer more immediate influences and shorter-term predictions. By analyzing the interaction between the dashas and transits, astrologers can make predictions, provide guidance, and suggest remedies to navigate through life's challenges and make the most of favorable periods.