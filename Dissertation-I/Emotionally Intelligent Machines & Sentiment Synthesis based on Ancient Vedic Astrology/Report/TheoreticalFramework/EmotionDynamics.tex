In 2017, Jon Miles published a blog on his website. In this blog named emotion dynamics or equation of emotions, he discussed about the mathematical modelling of human sentiments and different states of human mind which is based on the analogy of human mind with an electrical dynamical system. According to his blog, the three phenomenas of mind, craving, anger and attachment are related to each other as like the inductance, resistance and capacitance in an electrical network as shown in figure \ref{Fig:fig2} by a second order ordinary linear differential equation \ref{eq:3}.
\begin{figure}[H]
	\begin{center}
		\begin{circuitikz}
			\draw (0,0) to[short] (0,2);
			\draw (0,2) to[L=$L(Craving)$, i>_=$i(t)$] (3,2);
			\draw (3,2) to[R=$R(Anger)$, i>_=$i(t)$] (6,2);
			\draw (6,2) to[C=$C(Attachment)$, i>_=$i(t)$] (9,2);
			\draw (9,2) to[short] (9,0);
			\draw (9,0) to[vsource, v=$v(t)(Will Power)$] (0,0);
		\end{circuitikz}
	\end{center}
	\caption{Analogy of Emotion Dynamics with an Electrical System}
	\label{Fig:fig2}
\end{figure}
\vspace{1\baselineskip}

\begin{equation}\label{eq:3}
	\boxed{v(t) = L\frac{di(t)}{dt} + Ri(t) + \frac{1}{C}\int i(t)dt}
\end{equation}
\vspace{1\baselineskip}

If we convert this electrical system into its frequency domain equivalent by using Continous Time Fourier Transform, then equation \ref{eq:3} will be transformed into an algebraic equation \ref{eq:4}. According to the miles blog, the $ R $, $ X_{L} $ and $ X_{C} $ in equation \ref{eq:4} represents the habit, denial and doubt\cite{milesresearch}.
\begin{equation}\label{eq:4}
	\boxed{Z = R + j(X_{L}-X_{C})}
\end{equation}
where, $ R = $ Emotive Resistance,\\
$ X_{L} = $ Emotive Inductive Reactance,\\
$ X_{C} = $ Emotive Capacitive Reactance,\\
and $ j = \sqrt{-1} $ (Imaginary Unit)
\vspace{1\baselineskip}

On the other hand, in 2023, Deepanshu Giri, one of the famous astrologer of India discussed in an article of his blog about the analogy between some concepts of an electrical circuit like source, sink, active and passive with different kinds of energies associated with the nine planets in Vedic astrology. According to him, the active energy planets are Mars, Mercury and Venus whereas passive energy planets are Saturn and Jupiter. Also, he mentioned that the Sun and Moon are the continuous sources of energy combustion while Rahu and Ketu are the continuous sources of consumption in terms of electrical source and sink connection types\cite{DeepanshuGiriBlog}.