Going beyond sentiment analysis or developing such AI systems who have their own sentiments is currently not studied. However, several examples can be found where different methods are used to develop such systems that are capable of responding emotionally after analysing the sentiments from the input, such as the Emotional Speech Synthesis News Dialogue System, GAN Human Emotion Synthesis, and Emotional Chatting Machine which are discussed in this seminar, but working of all these Emotional Artificial Intelligence(EAI) systems which are present currently is based on the logic not on the intution. If it may possible for us to make such an EAI systems which can works on the basis of the logic as well as on intution, then the huge gaps and mismatchs which are currently present inbetween the interaction of human and computers can be fulfilled. The AI researchers have focused on giving machines linguistic and mathematical-logical reasoning abilities, modelled after the classic linguistic and mathematical-logical intelligences. This seminar discusses about the new research that is giving machines skills of emotional intelligence. Machines have long been able to appear as if they have emotional feelings, but they are now being programmed to also learn when and how to display emotion in ways that enable them to appear empathetic or otherwise emotionally intelligent. They are now being given the ability to sense and recognize expressions of human emotion such as interest, distress, and pleasure, with the recognition that such communication is vital for helping them choose more helpful and less-aggravating behaviour. Emotionally Intelligent Machines are the systems that can recognize, interpret, process, and simulate human emotions which could be based on the concept of ancient vedic astrology. They are the machines which can adapt different situations and knows how to handle these situations more intelligently and smartly. This paper also highlights different aspects of emotional behaviour of human beings.