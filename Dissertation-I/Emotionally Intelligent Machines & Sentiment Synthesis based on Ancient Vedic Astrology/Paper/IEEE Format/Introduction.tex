Mention Research Question Here

Going beyond sentiment analysis or developing such AI systems who have their own sentiments is currently not studied. However, several examples can be found where different methods are used to develop such systems that are capable of responding emotionally after analysing the sentiments from the input, such as the \\\\\\\\
but working of all these EAI systems which are present currently is based on the logic not on the intution. If it may possible for us to make such an EAI system which works on the basis of the logic as well as on the intution also, then the huge gaps and mismatches which are currently present in between interaction of the human and computers can be fulfilled.

For half a century, artificial-intelligence researchers have focused on giving machines linguistic and mathematical-logical reasoning abilities, modelled after the classic linguistic and mathematical-logical intelligences. This paper describes new research that is giving machines skills of emotional intelligence. Machines have long been able to appear as if they have emotional feelings, but they are now being programmed to also learn when and how to display emotion in ways that enable them to appear empathetic or otherwise emotionally intelligent. They are now being given the ability to sense and recognize expressions of human emotion such as interest, distress, and pleasure, with the recognition that such communication is vital for helping them choose more helpful and less-aggravating behaviour.

Emotionally Intelligent Machines are the systems that can recognize, interpret, process, and simulate human emotions which could be based on the concept of the ancient vedic astrology. They are the machines which can adapt different situations and knows how to handle these situations more intelligently and smartly.

This paper also highlights different aspects of emotional behaviour of human beings.

We also suggested a hypothetical model of an Emotionally Intelligent Machine (EIM).