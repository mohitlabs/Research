\subsection{What is Artificial Intelligence?}
\subsection{Limitations of Artificial Intelligence}
\subsection{What is Emotional Intelligence?}
Emotional Intelligence (EI) refers to the ability to recognize, understand and manage one's own emotions as well as the emotions of others. It involves being able to use emotional information to guide thinking and behavior, and to navigate social situations effectively.

EI is often described as having four components: self-awareness, self-management, social awareness, and relationship management. Self-awareness involves recognizing and understanding one's own emotions, strengths, and weaknesses. Self-management involves being able to regulate one's own emotions and behaviors in response to different situations. Social awareness involves recognizing and understanding the emotions of others, as well as the social norms and expectations of different situations. Relationship management involves using emotional information to communicate effectively, build and maintain relationships, and resolve conflicts.

EI is considered an important factor in personal and professional success, as it can help individuals navigate social interactions, build strong relationships, and manage stress and challenges effectively.
\subsection{Emotional Artificial Intelligence}
\subsubsection{Definition}
\subsubsection{Limitations of Artificial Intelligence}
\subsubsection{Need, Importance \& Benifits}
\subsubsection{Applications}
\cite{ISSN-2456-2165}
\subsection{Conscious, Subconscious \& Unconscious of Aritificial Intelligence}
\subsection{Convergence, Divergence and Belief Systems of AI}
\subsubsection{Stability and Unstability}
\subsection{Emotion Dynamics}
\subsection{The Butterfly Effect \& Chaos Theory}
Richard A. Anthes in 2022 by his paper "Predictability \& Predictions" showed his experiences with predictability theory and weather predictions began as an undergraduate student at the University of Wisconsin in Madison in the early 1960s. His interest in numerical simulations led to the development of a simple nonlinear one-dimensional gravity wave model and later a nonlinear, baroclinic, three-dimensional model of the tropical cyclone. His experiences highlighted the challenges of numerical and physical instabilities in weather prediction models \cite{atmos13081292}. \cite{encyclopedia2030084}
\subsection{Quantum Level V/S Cosmic Level}
\subsubsection{Microscopic V/S Macroscopic}
\subsection{Plutchik's Wheel of Emotions}
\subsection{Ancient Vedic Astrology}
\subsubsection{Classification of Vedic Astrology}
\subsubsection{Surya Siddhanta}
\subsubsection{Vrihat Samhita}
\subsubsection{Vrihat Parashar Hora Shastra}
\subsubsection{Significance of Planets}