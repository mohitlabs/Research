Going beyond sentiment analysis or developing such AI systems who have their own sentiments is currently not studied. However, several examples can be found where different methods are used to develop such systems that are capable of responding emotionally after analysing the sentiments from the input, such as the \\\\\\\\
but working of all these EAI systems which are present currently is based on the logic not on the intution. If it may possible for us to make such an EAI system which works on the basis of the logic as well as on the intution also, then the huge gaps and mismatches which are currently present in between interaction of the human and computers can be fulfilled.

\subsection{What is Artificial Intelligence?}
Artificial Intelligence (AI) refers to the simulation of human intelligence in machines that are programmed to perform tasks that normally require human intelligence such as learning, problem-solving, decision-making, perception, language understanding, and more\cite{ISSN-2456-2165}. AI systems use algorithms and statistical models to analyze data, recognize patterns, and make predictions, without explicit instructions from human operators.

\subsection{Limitations of Artificial Intelligence}
Although artificial intelligence (AI) has made significant advancements in recent years, there are still some limitations to the technology. Some of the limitations are:
\begin{itemize}
	\item \textbf{Lack of Common Sense:} AI systems lack the common sense that humans have, which can make it difficult for them to understand complex situations and make appropriate decisions.
	\item \textbf{Limited Creativity:} AI systems are designed to operate within the parameters set by their algorithms and data, which limits their ability to generate truly creative solutions or ideas.
	\item \textbf{Lack of Emotional Intelligence:} AI systems are not capable of experiencing emotions, which limits their ability to understand and respond to emotional cues in human interactions.
\end{itemize}
Limitations of AI highlight the need for continued research and development to address these issues and improve the capabilities of these systems.

\subsection{What is Emotional Intelligence?}
Emotional Intelligence (EI) refers to the ability to recognize, understand and manage one's own emotions as well as the emotions of others. It involves being able to use emotional information to guide thinking and behavior, and to navigate social situations effectively\cite{ISSN-2456-2165}.

EI is often described as having four components: self-awareness, self-management, social awareness, and relationship management. Self-awareness involves recognizing and understanding one's own emotions, strengths, and weaknesses. Self-management involves being able to regulate one's own emotions and behaviors in response to different situations. Social awareness involves recognizing and understanding the emotions of others, as well as the social norms and expectations of different situations. Relationship management involves using emotional information to communicate effectively, build and maintain relationships, and resolve conflicts.

EI is considered an important factor in personal and professional success, as it can help individuals navigate social interactions, build strong relationships, and manage stress and challenges effectively.

\subsection{Emotional Artificial Intelligence}
Emotional Artificial Intelligence (EAI) is the ability of AI systems to recognize, understand, and respond appropriately to human emotions. It is an emerging field of AI that focuses on building machines that can perceive, interpret, and express emotions similar to human beings. 

Emotional AI uses various techniques such as natural language processing, sentiment analysis, and facial recognition to detect emotions in human interactions. These techniques are then used to train algorithms and models that can predict and respond to human emotions in real-time.

Emotional AI has numerous potential applications, such as improving customer service, enhancing human-robot interactions, and providing mental health support.

\subsection{Why Truly Intelligent Machines Need Emotions?}
Many machines are in our household items such as kitchen, bedroom, which are artificially intelligent to help us with our daily tasks, however, they are emotionally unintelligent to adapt to our fulfillment. If one desires an Artificial Intelligence, the Artificial Intelligence should be able to adapt to the individual's state of mind. At the present time, many leading companies have expanded the idea of Emotional Artificial Intelligence into their AI systems.\cite{ISSN-2456-2165}.

The need of EIM's are can be seen due to their numerous applications which are expanding rapidly.
Some of the common applications of EIM's include:
\begin{itemize}
	\item \textbf{Social Media:} They can be used in social media to analyse user's emotions and provide more personalized content.
	\item \textbf{Business Intelligence(BI) \& Operational Research(OR):} EIM's are more intelligent than the traditional machines as a result they can help in Decision Making which is involved in Business Intelligence \& Operational Research.
	\item \textbf{Human Resources:} In human resources to analyse employee's emotions and improve the work environment and productivity.
	\item \textbf{Development of Machine Ethics \& Computational Morality:} Emotional AI can play a vital role in the development of machine ethics and morality. It will allow machines to understand and respond to human emotions, which is an important component of ethical and moral decision-making.
	\item \textbf{Human-Computer Interaction(HCI):} Emotional AI is extremely helpful in HCI, as it enables computers to understand and respond to human emotions, making the interaction more natural, intuitive, and empathetic. Here are some ways in which Emotional AI can be useful in developing machine ethics and morality:
	\begin{itemize}
		\item \textbf{Understanding Human Emotions:} EAI can help machines to understand human emotions, which is an important component of ethical decision-making. For example, a machine that can detect when a human is experiencing fear or pain could adjust its behavior accordingly to avoid causing harm.
		\item \textbf{Ethical Decision-Making:} EAI can help machines to make more ethical decisions by taking into account human emotions and responses. For example, a self-driving car that can detect when a passenger is feeling anxious or stressed could adjust its driving style to provide a safer and more comfortable ride.
		\item \textbf{Morality and Empathy:} Emotional AI can help machines to exhibit more empathy towards humans, which is an important component of moral decision-making. For example, a robot that can detect when a human is feeling sad or lonely could provide comfort or companionship.
		\item \textbf{Human-Machine Collaboration:} Emotional AI can help facilitate collaboration between humans and machines by allowing machines to understand and respond to human emotions. This could lead to more effective and productive collaborations, as well as greater trust between humans and machines.
	\end{itemize}
	\item \textbf{Customer Service:} EAI can be used in customer service to understand customer's emotions and respond accordingly, improving customer satisfaction. It can also be used in getting customer reviews, feedbacks \& and conducting surveys.
	\item \textbf{Healthcare:} EAI could be very useful in healthcare sector to detect patient's emotions and provide appropriate treatment and care. They could be a great blessing for the treatment of psychic patients and for counselling of persons suffering form depression \& anxiety or even having suicidal tendency.
	\item \textbf{Education:} EAI can be used in education sector to improve the effectiveness of teaching by understanding the emotional state of the students and adapting the teaching method accordingly.
	\item \textbf{Marketing, Sales \& Advertisement:} EAI can be used in marketing to analyze customer's emotions and tailor marketing messages to maximize their impact which leads in increment of sales conversion rate.
	\item \textbf{Art \& Culture:} AI which can generate creative artistic content like Melodies \& Progressions in Music, Paintings \& Poetry is currently based on logical reasoning. By the development of Emotional AI, generation of this type of content can reach the next level of the arts which can also be used by the artists as a reference for their work.
	\item \textbf{Media \& Communication:} EAI has a huge potential to do great in the field of media as it can be used as a NEWS anchor or an interactive agent which can communicate with their listeners emoitionally.
	\item \textbf{Entertainment:} Emotionally Intelligent Machines can be used in the entertainment industry to create more immersive experiences for users by understanding their emotional responses, in gaming industry to create more engaging games that respond to the player's emotions also create the dynamic gaming environment accordingly, in movies industry to create ambience, environments and also in writing scripts.
\end{itemize}
\subsection{Convergence, Divergence and Belief Systems of Human Mind}
Belief system is a topic on which the more we talk, the less the words. This is what's the only reason due to which similarities and differences exists between the thoughts and mentalities of different people. This is what is responsible for influencing, changing, and strengthening our emotions.

Believing in something is called a belief system, and not believing in something is also a belief system in itself. Apart from this, if someone says 'I believe' and another says 'I don't believe,' one thing is common in both statements, and that is both the persons are in doubtful state that's why they are using the word 'Belive'. Neither of the two persons will ever say, 'I believe that I have two hands' or 'I believe that the Taj Mahal is located in Agra' because they know that these two things exists in reality. The word 'Belive' is always be used in the doubtful situations. The word 'believe' is used to justify whether a doubtful statement is a truth or a lie.

Now, we try to understand the belief system from a new perspective by momentarily setting aside all the discussions we have had so far about beliefs and belief systems. Consider the equation \ref{eq:1}, it is the example of a geometric series that converges absolutely. This infinite series is related to some philosophical questions considered in antiquity, particularly to Zeno's paradoxes\cite{enwiki:1123380424}.
\[1 = \dfrac{1}{2} + \dfrac{1}{4} + \dfrac{1}{8} + \dfrac{1}{16} + \dfrac{1}{32} + ...\]
\begin{equation}\label{eq:1}
	\implies\boxed{1 = \sum\limits_{n = 1}^\infty\dfrac{1}{2^n}}
\end{equation}

Now, consider the equation \ref{eq:2} which is an infinite divergent series also known as the Ramanujan summation. Ramanujan summation is a technique invented by the mathematician Srinivasa Ramanujan for assigning a value to divergent infinite series. Although the Ramanujan summation of a divergent series is not a sum in the traditional sense, it has properties that make it mathematically useful in the study of divergent infinite series, for which conventional summation is undefined\cite{enwiki:1148245424}. And in mathematics, whenever a value is undefined. Then it means that we can consider or assign any value on it's place. However, in such situations, it is best to consider a particular value that seems to be in accordance with the situation, without considering any other value, which is the most appropriate approach.
\[-\dfrac{1}{12} = 1 + 2 + 3 + 4 + 5 + ...\]
\begin{equation}\label{eq:2}
	\implies\boxed{-\dfrac{1}{12} = \sum\limits_{n = 1}^\infty n}
\end{equation}

\subsection{Conscious, Subconscious \& Unconscious of Aritificial Intelligence}
Human mind is one of the most important part of the entire human body which contains thoughts, imagination, memory, will power \& sensation. Every human being has it's own personality. Some have similar personalities, some have not. The individual's mindset is responsible for it's own personality and behaviour.

According to psychology, human mind is classified majorly into three categories. Conscious, subconscious \& unconscious mind. All the events around us which we are experiencing at the current instant which is also known as the "awareness", comes due to the conscious mind whereas all of our habits and routines which are formed due to the repetition of different task and our experiences are stored in the subconscious. 

The third type of category of mind is the most mysterious and powerful which is the unconscious mind. It operates beyond our conscious awareness. It is the part of our mind which contains thoughts, memories and emotions that we are not aware of, but that still influence our behaviour and feelings drastically. Conscious mind contains some short term memory in a very less amount. The content stored inside this type of mind can be changed easily. Subconscious mind has more long term memory than the conscious mind. It can store thoughts longer than the conscious mind which are difficult to change and involves practising something continuously, developing habits by doing continuous efforts, etc. in order to change the mindset. The unconscious mind has permanent memory which is almost impossible to be changed by any type of effort. It is the primary source of the human behaviour and personality. It is just like the default personality of any particular human being\cite{article}.
\subsubsection{Recurrent Neural Networks as Consciousness of AI}
\subsubsection{Feed Forward Networks as Subconscious of AI}
\subsubsection{How to design the Unconscious of AI?}
\cite{article}
\subsection{Emotion Dynamics}
\begin{center}
	\begin{circuitikz}
		\draw (0,0) to[short] (0,2);
		\draw (0,2) to[L=$L(Craving)$, i>_=$i(t)$] (3,2);
		\draw (3,2) to[R=$R(Anger)$, i>_=$i(t)$] (6,2);
		\draw (6,2) to[C=$C(Attachment)$, i>_=$i(t)$] (9,2);
		\draw (9,2) to[short] (9,0);
		\draw (9,0) to[vsource, v=$v(t)(Will Power)$] (0,0);
	\end{circuitikz}
\end{center}
\begin{equation}
	\boxed{v(t) = L\frac{di(t)}{dt} + Ri(t) + \frac{1}{C}\int i(t)dt}
\end{equation}
\begin{equation}
	\boxed{F[n] = F[n-1] + F[n-2]}
\end{equation}
\subsection{The Butterfly Effect \& Chaos Theory}
Richard A. Anthes in 2022 by his paper "Predictability \& Predictions" showed his experiences with predictability theory and weather predictions began as an undergraduate student at the University of Wisconsin in Madison in the early 1960s. His interest in numerical simulations led to the development of a simple nonlinear one-dimensional gravity wave model and later a nonlinear, baroclinic, three-dimensional model of the tropical cyclone. His experiences highlighted the challenges of numerical and physical instabilities in weather prediction models \cite{atmos13081292}. \cite{encyclopedia2030084}
In chaos theory, the butterfly effect is the sensitive dependence on initial conditions in which a small change in one state of a deterministic nonlinear system can result in large differences in a later state.
\subsection{Quantum Level V/S Cosmic Level}
\subsubsection{Microscopic V/S Macroscopic}
\subsection{Plutchik's Wheel of Emotions}
Plutchik proposed a psychoevolutionary classification approach for general emotional responses.[2][3] He considered there to be eight primary emotions—anger, fear, sadness, disgust, surprise, anticipation, trust, and joy.
He also created a wheel of emotions to illustrate different emotions. Plutchik first proposed his cone-shaped model (3D) or the wheel model (2D) in 1980 to describe how emotions were related \cite{enwiki:1136521972}.
\begin{figure}[H]
	\includegraphics[width=\columnwidth, keepaspectratio]{Plutchik'sWheelofEmotions.png}
	\caption{Plutchik's Wheel of Emotions}
	\label{Fig:fig1}
\end{figure}
Figure \ref{Fig:fig1} shows PWOE.
\subsection{Ancient Vedic Astrology}
A Brief History of Ancient Vedic Astrology
\subsubsection{Classification of Vedic Astrology}
\subsubsection{Surya Siddhanta}
\subsubsection{Vrihat Samhita}
\subsubsection{Brihat Parashar Hora Shastra}
101 Chapters
4500 Verses
\subsubsection{Significance of Planets}
\begin{sanskrit}
	\begin{center}
		देवेज्यो ज्ञानसुखदो भृगुर्वीर्यप्रदयकः।\\ऋषिभिः प्राक्‌तनैः प्रोक्तश्छायासूनुश्च दुःखदः॥३:१४॥\cite{BrihatParasharHoraShastraVol1}
	\end{center}
\end{sanskrit}