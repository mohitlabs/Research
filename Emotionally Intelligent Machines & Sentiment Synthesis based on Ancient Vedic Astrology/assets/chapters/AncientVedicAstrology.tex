A Brief History of Ancient Vedic Astrology
\subsubsection{Classification of Vedic Astrology}
\subsubsection{Surya Siddhanta}
\subsubsection{Vrihat Samhita}
\subsubsection{Brihat Parashar Hora Shastra}
101 Chapters
4500 Verses
\subsubsection{Significance of Planets}
\begin{sanskrit}
	\begin{center}
		देवेज्यो ज्ञानसुखदो भृगुर्वीर्यप्रदयकः।\\ऋषिभिः प्राक्‌तनैः प्रोक्तश्छायासूनुश्च दुःखदः॥३:१४॥\cite{BrihatParasharHoraShastraVol1}
	\end{center}
\end{sanskrit}
The planetary vibrations reflected or refracted along with solar radiations to the earth are of
varying intensities as per planetary distance, size, and movement in the solar system. These
vibrations impact our sensory nerves, mental attitudes, and moods. Thus, it’s very likely that
these planetary vibrations supply the energies to the body cells though our nerves. Since
these vibrations differ in wavelength intensity and frequency as per the planetary properties
and motion; these vibrations supply different sensory stimuli which impacts the human
unconscious and personality at the time of birth\cite{article}.
\subsubsection{Birth Chart (An Initial Condition of Emotion Dynamics)}
The chart representing the planetary positions at the time of the of a child which observed from the place of birth is said as the birth chart of that person. It is like a snapshot of the universe at the time of birth of the individual.
If we consider the birth chart of an individual as an initial condition for it's neurodynamical system, then we can predict many things about the future behaviour of that particular individual. Apart from that, for simplicity and for the purpose of sentiment synthesis in this paper, we are only interested about the emotion dynamics of that person.
\subsubsection{Non-Linearity \& Time Variance}
Time-Varying Neural Networks