In 2023, Martins, Paulo in their paper ``A Concise History of Hindu Astrology and Indian Spirituality" provides a concise history of Hindu astrology and its symbolism present in its spirituality which is transversal to all cultures. They also mentioned the foundations and main conceptions of Hindu astrology, namely Rasi, Nakshastra, Navagrahas, Bhava, Dashas\cite{article2}.

In 2023, Bhardwaj, Rishi and Pareek, Aditya in their paper ``The Unconscious Mind and Planetary Influences on the Human Unconscious Mind and Personality" discusses the concept of the unconscious mind and its importance in psychology. The authors argue that the unconscious mind is not separate from the rest of the universe and that there is a connection between the human unconscious and the universal unconscious. They also discuss the impact of planetary vibrations on the huma physiology as well as on human psychology\cite{article1}.

In 2022, Anthes, Richard A. in their paper ``Predictability and Predictions" describes their experiences with predictability theory and weather predictions. The author classified the development of mesoscale weather systems into two types: those resulting from forcing by surface inhomogeneities and those resulting from internal modifications of large-scale flow patterns. The author also developed a nonlinear, baroclinic, three-dimensional model of the tropical cyclone and suffered through various forms of numerical and physical instabilities. The numerical instabilities could be controlled by suitable choices of finite difference schemes and various damping or smoothing mechanisms, but physical instabilities persisted and resulted in the evolution of somewhat realistic mesoscale features such as rainbands and eddies on the outflow layer that were not present in the initial conditions\cite{atmos13081292}.

In 2022, Shen, Bo-Wen and Pielke, Roger A. and Zeng, Xubin and Cui, Jialin and Faghih-Naini, Sara and Paxson, Wei and Atlas, Robert in their paper ``Three Kinds of Butterfly Effects within Lorenz Models" discussed about the three major kinds of butterfly effects within Lorenz models: (1) butterfly effects of the first kind (BE1) represent the sensitive dependence of solutions on initial conditions (SDIC); (2) butterfly effects of the second kind (BE2) represent the hypothetical role of initial tiny perturbations in producing an organized large-scale system at large distances; and (3) butterfly effects of the third kind (BE3), or the so-called real butterfly effect, represent the role of small scale processes in contributing to the finite predictability of large scale processes. The paper also provides a brief summary of the three kinds of butterfly effects and their differences. Additionally, the paper discusses the features of classical Lorenz models and a generalized Lorenz model\cite{encyclopedia2030084}.

In 2021, Rui Liu and Berrak Sisman and Haizhou Li in their paper ``Reinforcement Learning for Emotional Text-to-Speech Synthesis with Improved Emotion Discriminability" proposes a new interactive training paradigm for Emotional Text-to-Speech Synthesis (ETTS) called i-ETTS, which aims to improve the emotion discriminability of the generated voice by interacting with a Speech Emotion Recognition (SER) model. The proposed i-ETTS outperforms the state-of-the-art baselines by rendering speech with more accurate emotion style. The authors formulate an iterative training strategy with reinforcement learning to ensure the quality of i-ETTS optimization. The proposed i-ETTS achieves remarkable performance by consistently outperforming the ETTS baseline systems in terms of voice quality and emotion discriminability\cite{liu21p_interspeech}.

In 2020, Takatsu, Hiroaki  and Ando, Ryota and Matsuyama, Yoichi  and Kobayashi, Tetsunori in their paper ``Sentiment Analysis for Emotional Speech Synthesis in a News Dialogue System" proposes a method to control emotional parameters of speech synthesis in a news dialogue system by constructing a news dataset with emotion labels annotated for each sentence. They use a model combining BERT and BiLSTM-CRF to identify emotion labels and evaluate its effectiveness using the constructed dataset. The model performance can be improved by preferentially annotating articles with low confidence in the human-in-the-loop machine learning framework. The future work includes developing a speech synthesis system that can control emotional parameters using the emotion label estimated by the proposed model and confirming whether speaking with emotion promotes users' understanding in news delivery tasks\cite{takatsu-etal-2020-sentiment}.

In 2020, Dang, Nhan Cach and Moreno-García, María N. and De la Prieta, Fernando in their paper ``Sentiment Analysis Based on Deep Learning: A Comparative Study" discusses the use of deep learning models for sentiment analysis on social network data. The authors review the latest studies that have employed deep learning to solve sentiment analysis problems, such as sentiment polarity. They used word embedding and TF-IDF to transform input data before feeding that data into deep learning models. The architectures of DNN, CNN, and RNN were analyzed and combined with word embedding and TF-IDF to perform sentiment analysis. The authors conducted experiments to evaluate DNN, CNN, and RNN models on datasets of different topics, including tweets and reviews. Finally, a comparative study has been conducted on the experimental results obtained for the different models and input features\cite{electronics9030483}.

In 2020, Paul Clements in their paper ``Astrology, modernity and the project of self-identity" discusses Western and UK astrology as a fluid divinatory practice that accommodates modern, linear, and literal symbolism while still retaining its pre-modern 'magical' roots. It offers a spiritual understanding, self-knowledge, and self-determination, and encourages elective biography and self-identity. The practice of astrology today is a permutation of esoteric, individual DIY, and sun-sign formats, which offers multiple levels of engagement, from everyday meanings to more personal and philosophical insights. The astrologer mediates psychic hunches embedded in learnt craft, and it grounded some of the ideas presented, including the difficult choices surrounding individual definition and responsibility. The paper concludes that astrology embeds a spiritual outlook that co-exists with profane individualism and materiality highlighting dissonant modernity\cite{doi:10.1080/14755610.2022.2093234}.

In 2020, Hajarolasvadi, Noushin and Arjona Ramírez, Miguel and Demirel, Hasan in their paper ``Generative Adversarial Networks in Human Emotion Synthesis:A Review" reviews recent advances in human emotion synthesis using generative adversarial network (GAN) models. GAN models consist of a generator and a discriminator, which are trained iteratively in an adversarial learning manner, approaching Nash equilibrium. The core idea of GANs is based on a zero-sum game in game theory. Instead of estimating the distribution of real data samples, GANs learn to synthesize samples that adapt to the distribution of real data samples. The paper discusses facial expression synthesis, speech emotion synthesis, and audio-visual (cross-modal) emotion synthesis under different application scenarios. The authors also highlight open research problems to push the boundaries of this research area for future works\cite{article0}.

In 2019, Abbasi, Mohsin and Beltiukov, Anatoly in their paper ``Summarizing Emotions from Text Using Plutchik's Wheel of Emotions" discusses the analysis of emotions expressed by people on the internet using Plutchik's wheel of emotions. The wheel is used as a tool to identify and summarize emotions to their primary classes. The methodology involves allocating a weight to each emotion depending on the class it belongs to and its distance from the center of the wheel. These weights are then multiplied by the frequencies of emotions in text to identify their intensity level. The intensity of each emotion is summed up with the intensity of its primary emotion while summarizing it. The paper concludes that the methodology effectively summarizes emotions in the text, but neutral emotions and feelings described in Plutchik's wheel of emotion complicate the process of summarization. In future, the authors plan to propose a mechanism to avoid complications while summarizing neutral emotions\cite{article3}.

In 2019, Sahiti S. Magapu, Sashank Vaddiparty in their paper ``The Study of Emotional Intelligence in
Artificial Intelligence" discusses the role of Emotional Intelligence in Artificial Intelligence and its potential applications in various fields such as healthcare, education, consultation, and construction. The use of Emotional Artificial Intelligence can help machines to better understand and respond to human emotions, which can lead to more advanced solutions to complicated problems. It can also help to close the barriers between humans and machines, providing new opportunities for equal treatment. The conclusion of the paper is that the use of Emotional Artificial Intelligence gives a much more profound view on how machines can help humans compared to traditional AI today\cite{ISSN-2456-2165}.

In 2019, Piletsky, Eugene in their paper ``Consciousness and Unconsciousness of Artificial Intelligence" discusses the need to understand the problem of multilevel mind in artificial intelligence systems. It proposes that consciousness and the unconscious are not equal in natural mental processes and that the alleged mental activity of Artificial Intelligence may be devoid of the evolutionary characteristics of the human mind. The paper presents several scenarios for the possible development of a 'strong' AI through the prism of creation (or evolution) of the machine unconscious. It also proposes two opposite approaches regarding the relationship between the unconscious and the conscious. The conclusion raises interesting questions about whether a machine can have a phenomenal experience or something remotely resembling it, and whether there is a fundamental difference between the imitation of rational behavior and the rational behavior itself\cite{article4}.

In 2019, Chen, Ruiqi and Zhou, Yanquan and Zhang, Liujie and Duan, Xiuyu in their paper ``Word-level sentiment analysis with reinforcement learning" proposes a new framework named Word-level Sentiment LSTM (WS-LSTM) that uses reinforcement learning to realize text sentiment analysis. The framework uses three different LSTM tunnels for each action (Positive, Neutral, and Negative) to get sentiment tendency for each word in a sentence. The model can get word-level sentiment sequence with a relatively good result through reinforcement learning. The conclusion of the paper is that the proposed method can successfully combine text sentiment analysis with reinforcement learning and can get sentiment for each word in a specific task\cite{article5}.

In 2017, Sharma, Raksha and Somani, Arpan and Kumar, Lakshya and Bhattacharyya, Pushpak in their paper ``Sentiment Intensity Ranking among Adjectives Using Sentiment Bearing Word Embeddings" proposes a semi-supervised technique that uses sentiment bearing word embeddings to produce a continuous ranking among adjectives that share common semantics. The system demonstrates a strong Spearman's rank correlation of 0.83 with the gold standard ranking. The use of sentiment embeddings reduces the need for sentiment lexicon for identification of polarity orientation of words. Results show that Sentiment Specific Word Embeddings (SSWE) are significantly better than word2vec and GloVe, which do not capture sentiment information of words for intensity ranking task. The sentiment intensity information of words can be used in various NLP applications, for example, star-rating prediction, normalization of over-expressed or under-expressed texts, etc\cite{article7}.

In 2017, Zhou, Hao and Huang, Minlie and Zhang, Tianyang and Zhu, Xiaoyan and Liu, Bing in their paper ``Emotional Chatting Machine: Emotional Conversation Generation with Internal and External Memory" proposes Emotional Chatting Machine (ECM) that can generate appropriate responses not only in content but also in emotion. The model addresses the emotion factor using three new mechanisms that respectively (1) models the high-level abstraction of emotion expressions by embedding emotion categories, (2) captures the change of implicit internal emotion states, and (3) uses explicit emotion expressions with an external emotion vocabulary. The proposed model can generate responses appropriate not only in content but also in emotion, as shown by objective and manual evaluation\cite{article6}.

In 1997, Hochreiter, Sepp and Schmidhuber, Jürgen showed in their paper ``Long Short-term Memory" introduces a novel, efficient, gradient-based method called Long Short-Term Memory (LSTM) to solve long time lag problems. LSTM is local in space and time, its computational complexity per time step and weight is O(1). Each memory cell's internal architecture guarantees constant error flow within its constant error carrousel (CEC), provided that truncated backprop cuts off error flow trying to leak out of memory cells. Two gate units learn to open and close access to error flow within each memory cell's CEC. The multiplicative input gate affords protection of the CEC from perturbation by irrelevant inputs. Likewise, the multiplicative output gate protects other units from perturbation by currently irrelevant memory contents\cite{article}.