Belief system is a topic on which the more we talk, the less the words. This is what's the only reason due to which similarities and differences exists between the thoughts and mentalities of different people. This is what is responsible for influencing, changing, and strengthening our emotions.

Believing in something is called a belief system, and not believing in something is also a belief system in itself. Apart from this, if someone says `I believe' and another says `I don't believe,' one thing is common in both statements, and that is both the persons are in doubtful state that's why they are using the word `belive'. Neither of the two persons will ever say, `I believe that I have two hands' or `I believe that the Taj Mahal is located in Agra' because they know that these two things exists in reality. The word `belive' is always be used in the doubtful situations. It is used to justify whether a doubtful statement is a truth or a lie.

Now, we try to understand the belief system from a new perspective by momentarily setting aside all the discussions we have had so far about beliefs and belief systems. Consider the equation \ref{eq:1}, it is the example of a geometric series that converges absolutely. This infinite series is related to some philosophical questions considered in antiquity, particularly to Zeno's paradoxes\cite{enwiki:1123380424}.

\[1 = \dfrac{1}{2} + \dfrac{1}{4} + \dfrac{1}{8} + \dfrac{1}{16} + \dfrac{1}{32} + ...\]
\begin{equation}\label{eq:1}
	\implies\boxed{1 = \sum\limits_{n = 1}^\infty\dfrac{1}{2^n}}
\end{equation}

Now, consider the equation \ref{eq:2} which is an infinite divergent series also known as the Ramanujan summation. Ramanujan summation is a technique invented by the mathematician Srinivasa Ramanujan for assigning a value to divergent infinite series. Although the Ramanujan summation of a divergent series is not a sum in the traditional sense, it has properties that make it mathematically useful in the study of divergent infinite series, for which conventional summation is undefined\cite{enwiki:1148245424}. And in mathematics, whenever a value is undefined, then it means that we can consider or assign any value on it's place. However, in such situations, it is best to consider a special value that seems to be in accordance with the situation, without considering any other value, which is the most appropriate approach.

\[-\dfrac{1}{12} = 1 + 2 + 3 + 4 + 5 + ...\]
\begin{equation}\label{eq:2}
	\implies\boxed{-\dfrac{1}{12} = \sum\limits_{n = 1}^\infty n}
\end{equation}

The concept of convergence and divergence plays a very important role in the theory of dynamical systems because this is the concept by which the stability of a system is discussed. If the output of a system converges at a particular value, then the system is said to be a stable system, whereas if the output of a system diverges, then the system is said to be an unstable system. The behaviour of an unstable system is unexpected as it gives any random value on it's output.

Now, if we examine this mathematical concept of convergence and divergence carefully, then we can easily observe that our human mind also follows the same pattern, and works in a similar way. Whenever the biological neural networks present in our brain converges at some particular ouput, then only a believe about something is created in our mind. Contrary to this, if the output of the biological neural network present in the brain of a person diverges, then it will be a very dangerous situation because at that time, the mind of that individual will become unstable and behave unexpectedly as it can be observed in the cases of mental disorders.

Believe system plays a very significant role over the entire life of any individual. Whenever a new baby takes birth, its mind is like a blank sheet of paper. As the baby grows and interacts with its surrounding environment and situations, it begins to form beliefs about things in its mind, which we call its experience. Practically, we are able to identify, recognize and feel every smallest to the big things around us only due to our beliefs.